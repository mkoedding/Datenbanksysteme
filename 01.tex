\documentclass[12pt,a4paper,notitlepage,leqno]{article}

\author{Vorbereitungsseminar}
\newcommand{\thema}{Datenbankabfragen}










\usepackage[utf8]{inputenc}
\usepackage[german]{babel}
\usepackage[T1]{fontenc}
\usepackage{amsmath}
\usepackage{amsfonts}
\usepackage{amssymb}
\usepackage{graphicx}
\usepackage{blindtext}
\usepackage{hyperref}
\usepackage{commath}
\usepackage{minted}
\DeclareMathSymbol{*}{\mathbin}{symbols}{"01}
\usepackage{etoolbox}
\usepackage[left=2.5cm,right=2.5cm,top=2.5cm,bottom=2cm]{geometry}
\newcommand{\java}[1]{\mintinline[breaklines]{java}{#1}}
\renewcommand{\labelenumi}{\texttt{\alph{enumi}})}
\newcommand{\wenn}{&\text{, wenn }}
\newcommand{\sonst}{&\text{, sonst }}
\newcounter{Woche}
\setcounter{Woche}{0}
\newcounter{Aufgabe}
\setcounter{Aufgabe}{0}
\newcounter{KAufgabe}
\setcounter{KAufgabe}{0}



\newenvironment{aufgabe}[1]{
\addtocounter{Aufgabe}{1} \medskip%
\underline{\textbf{Aufgabe \theAufgabe}\;\;[#1]}

\vspace{0.3cm}}{\bigskip}

\newenvironment{klaufgabe}[2]{
\addtocounter{KAufgabe}{1} \medskip%
\underline{\textbf{Aufgabe \theKAufgabe}\;\;[#1]}\hfill\textit{(#2)}

\vspace{0.3cm}}{\newpage}

\newenvironment{paufgabe}[1]{
\addtocounter{Aufgabe}{1} \medskip%
\underline{\textbf{Aufgabe P.\theWoche.\theAufgabe}\;\;[#1]}

\vspace{0.3cm}}{\newpage}

\newcommand{\grundlagen}[2]{\textbf{Grundlagen} der Übung sind die Kapitel \textit{#1} bis \textit{#2}.}
\newcommand{\grundlage}[1]{\textbf{Grundlage} der Übung ist \textit{#1}.}
\newcommand{\hinweise}[1]{\textbf{Hinweise: } #1.}

\setlength{\parindent}{0em}

%\newcommand{\header}[3]{\begin{center}
%\setcounter{page}{1}
%%\noindent\rule[1ex]{\textwidth}{2pt}
%\begin{Large}
%\textbf{#1. #2 zum Thema\\#3}
%\end{Large}
%\end{center}
%Hr. Marvin Ködding \hfill \sj
%
%\vspace{0.05cm}
%\noindent\rule[1ex]{\textwidth}{2pt}
%\vspace{0.4cm}}

\newcommand{\pheader}[1]{
\setcounter{Aufgabe}{0}
\setcounter{Woche}{#1}
\vspace{0.2cm}
\begin{minipage}[c]{0.5\textwidth}
\begin{center}
\textsc{Fachbereich Mathematik, Informatik, Naturwissenschaften und Technik}
\end{center}
\end{minipage}
\begin{minipage}[c]{0.5\textwidth}
\begin{center}
\textbf{Datenbanksysteme}\\
Schuljahr 2022/23
\end{center}
\end{minipage}

\begin{center}
\textbf{\underline{Plenumsblatt \theWoche}}

\vspace{0.2cm}
\end{center}
}

\newcommand{\kheader}[1]{
\setcounter{KAufgabe}{0}

\vspace{0.2cm}
\begin{minipage}[c]{0.5\textwidth}
\begin{center}
\textsc{Fachbereich Mathematik, Informatik, Naturwissenschaften und Technik}
\end{center}
\end{minipage}
\begin{minipage}[c]{0.5\textwidth}
\begin{center}
\textbf{\thema}\\
Marvin Ködding
\end{center}
\end{minipage}

\begin{center}
\textbf{\underline{Klausur #1}}

\vspace{0.2cm}
\end{center}
\underline{Hinweise:}
\begin{itemize}
\item Schalten Sie, falls noch nicht geschehen, umgehend Ihr Mobiltelefon aus!
\item Schalten Sie außerdem alle nicht medizinisch notwendigen Lärmquellen aus.
\item Entfernen Sie jetzt alle unerlaubten Gegenstände vom Tisch. Erlaubt sind nur ein Stift (kein Rot-, Grün- oder Bleistift) und Getränke
\item Schreiben Sie auf jedes Blatt Ihren Namen. Blätter ohne Namen werden nicht korrigiert und ergeben 0 Punkte!
\item Verwenden Sie kein eigenes Papier für Notizen. Am Ende der Klausur befinden sich 2 Extrablätter. Sie können auf Anfrage weitere Blätter erhalten. Machen Sie gut kenntlich, wenn
Sie Zusatzblätter für Lösungen verwenden und tragen Sie dort ebenfalls Ihren Namen ein.
\item Die Bearbeitungszeit beträgt \textbf{90} Minuten
\item Mehrere, widersprüchliche Lösungen zu einer Aufgabe werden mit 0 Punkten bewertet.
\end{itemize}
}

\newcommand{\header}{
\setcounter{Aufgabe}{0}
\addtocounter{Woche}{1}
\vspace{0.2cm}
\begin{minipage}[c]{0.5\textwidth}
\begin{center}
\textsc{Fachbereich Mathematik, Informatik, Naturwissenschaften und Technik}
\end{center}
\end{minipage}
\begin{minipage}[c]{0.5\textwidth}
\begin{center}
\textbf{\thema}\\
Marvin Ködding
\end{center}
\end{minipage}

\begin{center}
\textbf{\underline{Übungsblatt \theWoche}}

\vspace{0.2cm}
\end{center}
\underline{Hinweise:}
\begin{itemize}
\item Die Aufgaben sind dem Skript von Michael Kipp (michaelkipp.de) entnommen.
\end{itemize}
}

\begin{document}

\vspace{0.2cm}
\begin{minipage}[c]{0.5\textwidth}
\begin{center}
\textsc{Fachbereich Mathematik, Informatik, Naturwissenschaften und Technik}
\end{center}
\end{minipage}
\begin{minipage}[c]{0.5\textwidth}
\begin{center}
\textbf{Datenbanksysteme}\\
Schuljahr 2022/23
\end{center}
\end{minipage}

\begin{center}
\textbf{\underline{Übungsblatt 1}}

\vspace{0.2cm}
\end{center}

\begin{aufgabe}{Modellierung I}
Es sei folgende Situation gegeben:\medskip

\textsf{Es gibt Bundesländer. Jedes Bundesland hat einen Namen und eine gewisse Zahl an Einwohnern. Zudem gibt es Städte und Flüsse. Städte haben einen Namen und eine Einwohnerzahl. Flüsse haben einen Namen. Eine Stadt kann Hauptstadt von einem Bundesland sein. Eine Stadt kann an einem Fluss liegen und Flüsse fließen durch ein Bundesland.}\medskip

Entwirf hierzu ein ER-Diagramm. Finde sinnvolle Kardinalitäten und Optionalitäten. Gib sinnvolle Schlüssel an.

\end{aufgabe}

\begin{aufgabe}{Modellierung II}
Es sei folgende Situation gegeben:\medskip

\textsf{In einer Schule gibt es Lehrer, die durch Vornamen und Nachnamen identifiziert werden. In dieser Schule gibt es verschiedene Fächer, die durch ein eindeutiges Kürzel (D,E,M) und eine lange Bezeichnung (Deutsch, Englisch, Mathematik) gekennzeichnet sind. Jeder Lehrer unterrichtet mehrere Fächer und jedes Fach wird von mehreren Lehrern unterrichtet. Ein Lehrer kann Fachleiter eines Faches sein und ein Fach braucht immer genau einen Fachleiter.}\medskip

Entwirf hierzu ein ER-Diagramm. Finde sinnvolle Kardinalitäten und Optionalitäten. Gib sinnvolle Schlüssel an.

\end{aufgabe}

\begin{aufgabe}{Modellierung III}
Es sei folgende Situation gegeben:\medskip

     {\sf Das Küchenstudio ''Musterküchen'' benötigt eine Datenbank, welche folgende Anforderungen erfüllt: Das Küchenstudio bietet verschiedene Küchenmöbel an (z.B. Hängeschrank ''Top Fred'', Händeschrank ''Herzog'', Spüle ''Superclean'', \dots). Jedes Küchenmöbel wird von genau einem Hersteller bezogen. Jedes Küchenmöbel hat bestimmte Abmessungen und gehört zu einer Kategorie (z.B. Hängeschrank, Spüle). Die Mitarbeiter des Küchenstudios verkaufen die Küchen. Küchen bestehen aus mehreren Küchenmöbeln. Ein Küchenmöbel kann in mehrere Küchen eingebaut werden. Für den Verkauf einer Küche ist jeweils ein Mitarbeiter zuständig.}\medskip

Entwirf hierzu ein ER-Diagramm. Finde sinnvolle Kardinalitäten und Optionalitäten. Überlege dir weitere Attribute zu den Entitäten und gib einen Schlüssel für jede Entität an.
\end{aufgabe}

\end{document}
