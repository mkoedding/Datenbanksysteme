\documentclass[12pt,a4paper,notitlepage,leqno]{article}

\author{Vorbereitungsseminar}
\newcommand{\thema}{Datenbankabfragen}










\usepackage[utf8]{inputenc}
\usepackage[german]{babel}
\usepackage[T1]{fontenc}
\usepackage{amsmath}
\usepackage{amsfonts}
\usepackage{amssymb}
\usepackage{graphicx}
\usepackage{blindtext}
\usepackage{hyperref}
\usepackage{commath}
\usepackage{minted}
\DeclareMathSymbol{*}{\mathbin}{symbols}{"01}
\usepackage{etoolbox}
\usepackage[left=2.5cm,right=2.5cm,top=2.5cm,bottom=2cm]{geometry}
\newcommand{\java}[1]{\mintinline[breaklines]{java}{#1}}
\renewcommand{\labelenumi}{\texttt{\alph{enumi}})}
\newcommand{\wenn}{&\text{, wenn }}
\newcommand{\sonst}{&\text{, sonst }}
\newcounter{Woche}
\setcounter{Woche}{0}
\newcounter{Aufgabe}
\setcounter{Aufgabe}{0}
\newcounter{KAufgabe}
\setcounter{KAufgabe}{0}



\newenvironment{aufgabe}[1]{
\addtocounter{Aufgabe}{1} \medskip%
\underline{\textbf{Aufgabe \theAufgabe}\;\;[#1]}

\vspace{0.3cm}}{\bigskip}

\newenvironment{klaufgabe}[2]{
\addtocounter{KAufgabe}{1} \medskip%
\underline{\textbf{Aufgabe \theKAufgabe}\;\;[#1]}\hfill\textit{(#2)}

\vspace{0.3cm}}{\newpage}

\newenvironment{paufgabe}[1]{
\addtocounter{Aufgabe}{1} \medskip%
\underline{\textbf{Aufgabe P.\theWoche.\theAufgabe}\;\;[#1]}

\vspace{0.3cm}}{\newpage}

\newcommand{\grundlagen}[2]{\textbf{Grundlagen} der Übung sind die Kapitel \textit{#1} bis \textit{#2}.}
\newcommand{\grundlage}[1]{\textbf{Grundlage} der Übung ist \textit{#1}.}
\newcommand{\hinweise}[1]{\textbf{Hinweise: } #1.}

\setlength{\parindent}{0em}

%\newcommand{\header}[3]{\begin{center}
%\setcounter{page}{1}
%%\noindent\rule[1ex]{\textwidth}{2pt}
%\begin{Large}
%\textbf{#1. #2 zum Thema\\#3}
%\end{Large}
%\end{center}
%Hr. Marvin Ködding \hfill \sj
%
%\vspace{0.05cm}
%\noindent\rule[1ex]{\textwidth}{2pt}
%\vspace{0.4cm}}

\newcommand{\pheader}[1]{
\setcounter{Aufgabe}{0}
\setcounter{Woche}{#1}
\vspace{0.2cm}
\begin{minipage}[c]{0.5\textwidth}
\begin{center}
\textsc{Fachbereich Mathematik, Informatik, Naturwissenschaften und Technik}
\end{center}
\end{minipage}
\begin{minipage}[c]{0.5\textwidth}
\begin{center}
\textbf{Datenbanksysteme}\\
Schuljahr 2022/23
\end{center}
\end{minipage}

\begin{center}
\textbf{\underline{Plenumsblatt \theWoche}}

\vspace{0.2cm}
\end{center}
}

\newcommand{\kheader}[1]{
\setcounter{KAufgabe}{0}

\vspace{0.2cm}
\begin{minipage}[c]{0.5\textwidth}
\begin{center}
\textsc{Fachbereich Mathematik, Informatik, Naturwissenschaften und Technik}
\end{center}
\end{minipage}
\begin{minipage}[c]{0.5\textwidth}
\begin{center}
\textbf{\thema}\\
Marvin Ködding
\end{center}
\end{minipage}

\begin{center}
\textbf{\underline{Klausur #1}}

\vspace{0.2cm}
\end{center}
\underline{Hinweise:}
\begin{itemize}
\item Schalten Sie, falls noch nicht geschehen, umgehend Ihr Mobiltelefon aus!
\item Schalten Sie außerdem alle nicht medizinisch notwendigen Lärmquellen aus.
\item Entfernen Sie jetzt alle unerlaubten Gegenstände vom Tisch. Erlaubt sind nur ein Stift (kein Rot-, Grün- oder Bleistift) und Getränke
\item Schreiben Sie auf jedes Blatt Ihren Namen. Blätter ohne Namen werden nicht korrigiert und ergeben 0 Punkte!
\item Verwenden Sie kein eigenes Papier für Notizen. Am Ende der Klausur befinden sich 2 Extrablätter. Sie können auf Anfrage weitere Blätter erhalten. Machen Sie gut kenntlich, wenn
Sie Zusatzblätter für Lösungen verwenden und tragen Sie dort ebenfalls Ihren Namen ein.
\item Die Bearbeitungszeit beträgt \textbf{90} Minuten
\item Mehrere, widersprüchliche Lösungen zu einer Aufgabe werden mit 0 Punkten bewertet.
\end{itemize}
}

\newcommand{\header}{
\setcounter{Aufgabe}{0}
\addtocounter{Woche}{1}
\vspace{0.2cm}
\begin{minipage}[c]{0.5\textwidth}
\begin{center}
\textsc{Fachbereich Mathematik, Informatik, Naturwissenschaften und Technik}
\end{center}
\end{minipage}
\begin{minipage}[c]{0.5\textwidth}
\begin{center}
\textbf{\thema}\\
Marvin Ködding
\end{center}
\end{minipage}

\begin{center}
\textbf{\underline{Übungsblatt \theWoche}}

\vspace{0.2cm}
\end{center}
\underline{Hinweise:}
\begin{itemize}
\item Die Aufgaben sind dem Skript von Michael Kipp (michaelkipp.de) entnommen.
\end{itemize}
}

\begin{document}

\vspace{0.2cm}
\pheader{1}

\begin{paufgabe}{Modellierung I}
Es sei folgende Situation gegeben:\medskip

{\sf Die Dienstleistungsgesellschaft für Informatik (Dfl-GmbH) ist eine Bildungseinrichtung im Raum Villingen-Schwenningen (VS). Das Unternehmen hat sich auf die Abnahme von IT-Zertifizierungen spezialisiert. Zu diesem Zweck unterhält sie an verschiedenen Standorten in Villingen-Schwenningen und Umgebung eine bestimmte Anzahl an PC-Räumen (z.B.: Volkshochschule VS, 4 Räume; Bildungszentrum Schwenningen, 2 Räume; InfoPlan-Center Villingen, 1 Raum u.a.). Jeder Standort wird von einem Mitarbeiter der Dfl-GmbH organisatorisch betreut.

Die Kunden der Dfl-GmbH können verschiedene IT-Zertifikate erwerben (z.B.: Verwaltung der Zertifizierungen soll mit dem Einsatz einer Relationalen Datenbank unterstützt werden.

Für den Erwerb eines Zertifikates melden sich die Teilnehmer zu einer Prüfung an, für die eine bestimmte Gebühr zu entrichten ist. Die Prüfungen werden jeweils an einem der Standorte der Dfl-GmbH durchgeführt. Eine Zertifikatsprüfung besteht in der Regel aus mehreren Prüfungsmodulen (''Betriebssysteme'', ''Textverarbeitung'', ''Tabellenkalkulation'' u.a.), die innerhalb eines Tages abgelegt werden müssen. Zu beachten ist, dass einige Prüfungsmodule Bestandteil mehrerer Zertifikatsprüfungen sein können. So ist beispielsweise das Prüfungsmodul ''Tabellenkalkulation'' sowohl für die Zertifikatsprüfung ''Open-OfficeSpezialist'' als auch für den ''IT-Experte'' relevant.
Für jede Prüfung sollen das Prüfungsdatum, das zu erwerbende Zertifikat, die Prüfungsgebühr sowie die angemeldeten Teilnehmer mit Vorname, Nachname, E-Mail-Adresse sowie das Anmeldedatum erfasst werden.

Für die Durchführung der Prüfungen ist jeweils ein Mitarbeiter der Dfl-GmbH als Prüfungsleiter verantwortlich. Für die Mitarbeiter sind der Name sowie die Anschrift mit Straße, Postleitzahl und Wohnort zu speichern.
Von den Standorten sind die Bezeichnung, die Anschrift mit Strasse, Postleitzahl und Ort sowie die Anzahl der PC-Räume zu speichern.}\medskip

Entwirf hierzu ein ER-Diagramm.

\end{paufgabe}

\begin{paufgabe}{Modellierung II}
\begin{enumerate}
    \item Entwirf ER-Schemata für die folgenden Anwendungsbereiche. In Klammern sind potentiell relevante Entity-Typen aufgezählt:
    \begin{itemize}
    \item \textsc{Fußball-Liga} (Mannschaften, Trainer, Spiele, Manager, Spieler)
    \item \textsc{Familie} (Väter, Mütter, Kinder, Wohnungen)
    \item \textsc{Bank} (Filialen, Kunden, Geldautomaten, Konten, Transaktionen)
\end{itemize}
    \item Begründe deinen Entwurf
    \item Spezifiziere für die Entitäten Attribute
\end{enumerate}

\end{paufgabe}

\end{document}
